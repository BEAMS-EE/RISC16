\documentclass[12pt,a4paper]{extarticle}
\usepackage[utf8]{inputenc}
\usepackage[english]{babel}
\usepackage[T1]{fontenc}
\usepackage{amsmath}
\usepackage{amsfonts}
\usepackage{amssymb}
\usepackage{graphicx}
\usepackage[breaklinks=true,hidelinks]{hyperref}

% \def\labelitemi{\makebox[0pt][l]{$\square$}\raisebox{.15ex}{\hspace{0.1em}$\checkmark$}}
\def\labelitemii{$\rightarrow$}

\newcommand{\done}{\makebox[0pt][l]{$\square$}\raisebox{.15ex}{\hspace{0.1em}$\checkmark$}}%Note: the argument of \makebox is the width of the box as seen by the rest of the text. If set to 0pt, the content of the box (here a \square) will overlap with the text (here a \checkmark).
\newcommand{\notdone}{\makebox[1em][l]{$\square$}}


% \author{Quentin Delhaye}
% \date{October 25th, 2015}
\title{RISC16 Simulator Debugging}

\begin{document}
\maketitle


\begin{itemize}
  \item[\notdone] In the sequential version, writting instructions in the program memory is not enough.
  It must first be saved in ROM, then imported back and only then can be executed.
  \begin{itemize}
    \item See what happens when we try to run the simulation. Where does it take the information from ?
    \item See what happens when we import the ROM. Is it loaded somewhere?
  \end{itemize}

\end{itemize}


\end{document}
